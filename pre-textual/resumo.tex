\setlength{\absparsep}{18pt} % ajusta o espaçamento dos parágrafos do resumo
\begin{resumo}
A cada dia são criados mais prestadores de serviços no Brasil. Para essas empresas, é necessário utilizar um sistema de TI para gerenciar, melhorar e refinar os processos de gestão da empresa. Um módulo importante para esses sistemas é o que trata do gerenciamento de ordens de serviço. O objetivo do trabalho é apresentar o sistema de gestão de ordens de serviço para e empresa Rexton.

O sistema será criado para o ambiente mobile, onde é possível realizar esse gerenciamento utilizando a plataforma Android com o framework React-native e o banco de dados PostgreSQL. Como resultado, será criado um sistema que permite gerenciar ordens de serviço, controlar estoques de peças e clientes.
	
\textbf{Palavras-chave}: Javascript, desenvolvimento, banco de dados, postgreSQL, ordens de serviço.

\end{resumo}