% ----------------------------------------------------------
% Introdução (exemplo de capítulo sem numeração, mas presente no Sumário)
% ----------------------------------------------------------
\chapter*[INTRODUÇÃO]{INTRODUÇÃO}
\addcontentsline{toc}{chapter}{INTRODUÇÃO}
O trabalho tem como objetivo uma empresa do setor de terraplenagem chamada Rexton, que possui uma riqueza de informações e possui apenas um departamento para gerenciar, embora seja necessário criar uma ferramenta para gerenciar os serviços prestados. Da necessidade de organizar e controlar a quantidade de informações, surge a necessidade de construir um sistema eletrônico de gestão de ordens de serviço.

Atualmente, um grande número de serviços é gerado no dia a dia da empresa, portanto, é necessário criar relatórios contendo fluxo de caixa, histórico de ordens de serviço, podendo separar por status, como pendentes, cancelados e concluídos, para exemplo, além de controlar os principais equipamentos da empresa. Essas informações devem ser armazenadas por um período de tempo predeterminado.

Atualmente, os relatórios da empresa são feitos manualmente e armazenados em pastas e alocados em espaço físico, e muitas vezes acaba não sendo o caminho certo a seguir. A inconveniência será enorme ao eventualmente pesquisar ou pesquisar qualquer um desses registros, pois muitas vezes o arquivo acumulado acaba se perdendo, ou sua localização é demorada e ineficiente porque você tem que ir página por página.

A empresa aqui mencionada pode ser considerada como MEI (Microempreendedor), e possui somente um funcionário. Diante disso, a principal questão a ser discutida é: como controlar efetivamente todos os serviços e execuções?

O sistema de ordens de serviço foi projetado para ajudar a administração a lidar com os serviços prestados dentro da empresa, uma questão que afeta totalmente a Rexton.