\chapter{Fundamentação teórica}
Este capítulo apresenta os fundamentos teóricos dos principais tópicos abordados neste trabalho, como prestação de serviços, ordem de serviço e sistemas de informação.
\section{PRESTAÇÃO DE SERVIÇOS}
O \citeonline{SENAC} esclarece: "Um serviço é o resultado de pelo menos uma atividade realizada na interface do fornecedor com o cliente. Geralmente é intangível".
De acordo com \citeonline{Pandjiarjian}, os prestadores de serviços estão contribuindo cada vez mais para o crescimento econômico mundial, respondendo por um terço do comércio mundial e criando 11 milhões de empregos, o equivalente a 16\% dos trabalhadores do setor privado. Como \citeonline{Fitzsimmons} argumenta, isso ocorre porque as empresas especializadas em um determinado tipo de serviço podem fornecer esse serviço de forma mais barata e eficiente para empresas de manufatura. Como resultado, é cada vez mais comum que determinados serviços sejam prestados por empresas de serviços.

A qualidade dos serviços prestados é da maior importância, pois cada vez mais clientes procuram serviços e esperam sempre a qualidade desses serviços. Para atingir a qualidade do serviço, é necessário criar um ambiente especial dentro da empresa para prestar um serviço de excelência aos clientes. é o compromisso de todos os membros da organização. \cite{SENAC}

\section{ORDEM DE SERVIÇO}
De acordo com SAP (Systems Applications Products) \cite{douglas_da_silva_2020}, uma ordem de serviço é um contrato de curto prazo entre um provedor de serviço e um destinatário de serviço no qual um único serviço é especificado no pedido e o faturamento relacionado ao recurso é realizado. É uma solicitação para realizar uma atividade de serviço em um objeto de manutenção de uma empresa cliente em uma data específica. Além disso, de acordo com a SAP, as ordens de serviço são utilizadas como uma utilidade para monitorar as atividades de um determinado serviço prestado.

\section{SISTEMAS DE INFORMAÇÃO}
Segundo \citeonline{rezende-abreu}, sistemas de informação são todos os sistemas que geram informações para realizar ações e auxiliar a tomada de decisões. É comum as empresas automatizarem processos por meio de um ou mais sistemas de informação.

Segundo Rezende e Abreu, toda empresa moderna preocupada com sua sustentabilidade e competitividade deve também focar na execução e organização das atividades de planejamento estratégico, sistemas de informação e gestão de TI.